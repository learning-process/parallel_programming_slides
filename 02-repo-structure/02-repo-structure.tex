\documentclass{beamer}

% Theme choice
\usetheme{Madrid}

% Optional packages
\usepackage{graphicx} % For including images
\usepackage{amsmath}  % For math symbols and formulas
\usepackage{hyperref} % For hyperlinks
\usepackage{listings}
\usepackage{xcolor}
\usepackage{tikz}
\usepackage[T1]{fontenc}

\lstdefinestyle{CStyle}{
  language=C,                    % Set the language to C
  basicstyle=\ttfamily\footnotesize\linespread{0.9}\tiny, % Set font style and size
  keywordstyle=\color{blue},      % Color of keywords
  commentstyle=\color{gray},      % Color of comments
  stringstyle=\color{red},        % Color of strings
  showstringspaces=false,         % Do not mark spaces in strings
  breaklines=true,                % Enable line breaks at appropriate places
  breakatwhitespace=false,        % Break lines at any character, not just whitespace
  numbers=left,                   % Show line numbers on the left
  numberstyle=\tiny\color{gray},  % Style for line numbers
  tabsize=4,                      % Set tab width
  keepspaces=true,                % Keep indentation spaces
  frame=single,                   % Add a border around the code
  aboveskip=0pt,                  % Reduce space above the code block
  belowskip=0pt,                   % Reduce space below the code block
  xleftmargin=7.5pt,                      % Add left padding (approx. 2.8mm or 10px)
  xrightmargin=15pt,                      % Add left padding (approx. 2.8mm or 10px)
}

\AtBeginSection[]{
  \begin{frame}
    \centering
    \Huge\insertsection
  \end{frame}
}

% Title, author, date, and institute (optional)
\title[Parallel Programming. Repository structure]{Parallel Programming course. Repository structure}
\author{Obolenskiy Arseniy, Nesterov Alexander}
\institute{Nizhny Novgorod State University}

\date{\today} % or \date{Month Day, Year}

% Redefine the footline to display both the short title and the university name
\setbeamertemplate{footline}{
  \leavevmode%
  \hbox{%
    \begin{beamercolorbox}[wd=.45\paperwidth,ht=2.5ex,dp=1ex,leftskip=1em,center]{author in head/foot}%
        \usebeamerfont{author in head/foot}\insertshortinstitute % Displays the university name
    \end{beamercolorbox}%
    \begin{beamercolorbox}[wd=.45\paperwidth,ht=2.5ex,dp=1ex,leftskip=1em,center]{author in head/foot}%
      \usebeamerfont{author in head/foot}\insertshorttitle % Displays the short title
    \end{beamercolorbox}%
    \begin{beamercolorbox}[wd=.1\paperwidth,ht=2.5ex,dp=1ex,rightskip=1em,center]{author in head/foot}%
      \usebeamerfont{author in head/foot}\insertframenumber{} / \inserttotalframenumber
    \end{beamercolorbox}}%
  \vskip0pt%
}

\begin{document}

\begin{frame}
    \titlepage
\end{frame}

\begin{frame}{Contents}
    \tableofcontents
\end{frame}

\section{The introduction to the repository}

\begin{frame}[fragile]{Parallel programming technologies}
  \begin{itemize}
    \item \textbf{MPI}
    \item OpenMP
    \item TBB
    \item std::thread
  \end{itemize}
\end{frame}

\begin{frame}{Documentation}
  Follow the documentation for detailed instructions and examples:
  \begin{itemize}
    \item EN: \href{https://learning-process.github.io/parallel\_programming\_course/en/}{https://learning-process.github.io/parallel\_programming\_course/en/}
    \item RU: \href{https://learning-process.github.io/parallel\_programming\_course/ru/}{https://learning-process.github.io/parallel\_programming\_course/ru/}
  \end{itemize}
\end{frame}

\begin{frame}[fragile]{Local Setup and Build}
  \begin{itemize}
    \item Dev Container (recommended): VS Code + Docker + Dev Containers extension
    \item Manual prerequisites: Install CMake; install MPI/OpenMP per OS (see docs)
    \item Download submodules: \\
      \verb|git submodule update --init --recursive --depth=1|
    \item Configure build: \\
      \verb|mkdir build && cd build| \\[-2pt]
      \verb|cmake -D USE_FUNC_TESTS=ON -D USE_PERF_TESTS=ON -D CMAKE_BUILD_TYPE=Release ..|
    \item Build: \\
      \verb|cmake --build . --config Release --parallel|
  \end{itemize}
\end{frame}

\begin{frame}[fragile]{Docker and Dev Container}
  You may use Docker container for the convenience of development.
  \begin{itemize}
    \item VS Code flow (recommended):
      \begin{itemize}
        \item Open the repo in VS Code, install Dev Containers extension
        \item Use: \verb|Dev Containers: Reopen in Container|
        \item VS Code builds a container with GCC, CMake, MPI, OpenMP, tools
      \end{itemize}
    \item Prebuilt image (in case you do not use dev container or VS Code):
      \begin{itemize}
        \item Pull: \verb|docker pull ghcr.io/learning-process/ppc-ubuntu:1.1|
        \item Run: \verb|docker run -it ghcr.io/learning-process/ppc-ubuntu:1.1|
      \end{itemize}
    \item Build locally (Dockerfile: \texttt{docker/ubuntu.Dockerfile}):
      \begin{itemize}
        \item Build: \verb|docker build -f docker/ubuntu.Dockerfile -t ppc-dev .|
        \item Run: \verb|docker run -it ppc-dev|
        \item Mount repo (optional): \verb|-v "$(pwd)":/work -w /work|
      \end{itemize}
  \end{itemize}
\end{frame}

\begin{frame}{References}
  \begin{itemize}
    \item PPC Repository \href{https://github.com/learning-process/parallel\_programming\_course}{https://github.com/learning-process/parallel\_programming\_course}
    \item PPC Docker image \href{https://github.com/orgs/learning-process/packages/container/package/ppc-ubuntu}{https://github.com/orgs/learning-process/packages/container/package/ppc-ubuntu}
    \item VS Code Dev Containers Extension \href{https://marketplace.visualstudio.com/items?itemName=ms-vscode-remote.remote-containers}{https://marketplace.visualstudio.com/items?itemName=ms-vscode-remote.remote-containers}
  \end{itemize}
\end{frame}

\end{document}

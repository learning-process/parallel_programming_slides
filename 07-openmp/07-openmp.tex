\documentclass{beamer}

% Theme choice
\usetheme{Madrid}

% Optional packages
\usepackage{graphicx} % For including images
\usepackage{amsmath}  % For math symbols and formulas
\usepackage{hyperref} % For hyperlinks
\usepackage{tikz}     % For charts
\usepackage{listings}
\usepackage{xcolor}
\usepackage[T1]{fontenc}

\lstdefinestyle{CStyle}{
  language=C,                    % Set the language to C
  basicstyle=\ttfamily\footnotesize\linespread{0.9}\tiny, % Set font style and size
  keywordstyle=\color{blue},      % Color of keywords
  commentstyle=\color{gray},      % Color of comments
  stringstyle=\color{red},        % Color of strings
  showstringspaces=false,         % Do not mark spaces in strings
  breaklines=true,                % Enable line breaks at appropriate places
  breakatwhitespace=false,        % Break lines at any character, not just whitespace
  numbers=left,                   % Show line numbers on the left
  numberstyle=\tiny\color{gray},  % Style for line numbers
  tabsize=4,                      % Set tab width
  keepspaces=true,                % Keep indentation spaces
  frame=single,                   % Add a border around the code
  aboveskip=0pt,                  % Reduce space above the code block
  belowskip=0pt,                   % Reduce space below the code block
  xleftmargin=7.5pt,                      % Add left padding (approx. 2.8mm or 10px)
  xrightmargin=15pt,                      % Add left padding (approx. 2.8mm or 10px)
}

% Title, author, date, and institute (optional)
\title[Parallel Programming Course. OpenMP.]{Parallel Programming Course. \\OpenMP.}
\author{Obolenskiy Arseniy, Nesterov Alexander}
\institute{Nizhny Novgorod State University}

\date{\today} % or \date{Month Day, Year}

% Redefine the footline to display both the short title and the university name
\setbeamertemplate{footline}{
  \leavevmode%
  \hbox{%
    \begin{beamercolorbox}[wd=.45\paperwidth,ht=2.5ex,dp=1ex,leftskip=1em,center]{author in head/foot}%
        \usebeamerfont{author in head/foot}\insertshortinstitute % Displays the university name
    \end{beamercolorbox}%
    \begin{beamercolorbox}[wd=.45\paperwidth,ht=2.5ex,dp=1ex,leftskip=1em,center]{author in head/foot}%
      \usebeamerfont{author in head/foot}\insertshorttitle % Displays the short title
    \end{beamercolorbox}%
    \begin{beamercolorbox}[wd=.1\paperwidth,ht=2.5ex,dp=1ex,rightskip=1em,center]{author in head/foot}%
      \usebeamerfont{author in head/foot}\insertframenumber{} / \inserttotalframenumber
    \end{beamercolorbox}}%
  \vskip0pt%
}

\begin{document}

% Title slide
\begin{frame}
  \titlepage
\end{frame}

\begin{frame}{Today}
  \tableofcontents
\end{frame}

\section{Introduction to OpenMP}
\begin{frame}{What is OpenMP?}
  \begin{itemize}
    \item Brief overview
    \item Importance in parallel computing
    \item Use cases
  \end{itemize}
\end{frame}

\begin{frame}{What is OpenMP?}
  \begin{itemize}
    \item Open standard for parallel programming
    \item Supports multi-platform shared-memory multiprocessing
    \item Used in computational science, engineering, and simulations
  \end{itemize}
\end{frame}

\section{Hello World}
\begin{frame}[fragile]{Your First OpenMP Program}
  \lstset{style=CStyle}
  \begin{lstlisting}
#include <omp.h>
#include <stdio.h>

int main() {
  #pragma omp parallel
  {
    printf("Hello from thread %d\n", omp_get_thread_num());
  }
  return 0;
}
  \end{lstlisting}
\end{frame}

\section{Basic OpenMP Features}
\begin{frame}{OpenMP Features}
    \begin{itemize}
      \item Compiler directives
      \item Runtime library functions
      \item Environment variables
    \end{itemize}
\end{frame}

\begin{frame}{Basic OpenMP Features}
  OpenMP provides three primary mechanisms to express parallelism clearly and effectively:

   \begin{itemize}
    \item Compiler directives (\texttt{\#pragma omp})
    \item Runtime library functions
    \item Environment variables
  \end{itemize}
\end{frame}

\section{Compiler Directives and Clauses}
\begin{frame}[fragile]{Compiler Directives}
  Compiler directives guide the compiler to parallelize sections of code.

  General syntax:
  \begin{verbatim}
    #pragma omp directive [clauses]
  \end{verbatim}

  Commonly used directives:
  \begin{itemize}
    \item \texttt{parallel} — Creates parallel threads.
    \item \texttt{for} — Parallelizes loop iterations.
    \item \texttt{section} — Defines parallel sections.
  \end{itemize}

  Example:
  \lstset{style=CStyle}
  \begin{lstlisting}
#pragma omp parallel for
for(int i = 0; i < N; i++) {
  a[i] = b[i] + c[i];
}
  \end{lstlisting}
\end{frame}

\begin{frame}[fragile]{The \texttt{parallel} Directive}
  The \texttt{parallel} directive starts a parallel region executed by multiple threads.

  Syntax:
  \begin{verbatim}
    #pragma omp parallel [clauses]
    {
      // Code executed in parallel
    }
  \end{verbatim}

  Example:
  \lstset{style=CStyle}
  \begin{lstlisting}
#pragma omp parallel
{
  printf("Thread %d is running\n", omp_get_thread_num());
}
  \end{lstlisting}
\end{frame}

\begin{frame}[fragile]{The \texttt{for} Directive}
  The \texttt{for} directive parallelizes loops among threads.

  To take an effect it must be within an existing parallel region:

  Syntax:
  \begin{verbatim}
    #pragma omp for [clauses]
    for (init; condition; increment) {
      // Loop body
    }
  \end{verbatim}

  Example:
  \lstset{style=CStyle}
  \begin{lstlisting}
#pragma omp parallel
{
  #pragma omp for
  for (int i = 0; i < N; i++) {
    array[i] = compute(i);
  }
}
  \end{lstlisting}
\end{frame}

\begin{frame}[fragile]{The \texttt{parallel for} Directive}
  Combines the \texttt{parallel} and \texttt{for} directives, simplifying syntax.

  Syntax:
  \begin{verbatim}
    #pragma omp parallel for [clauses]
    for (init; condition; increment) {
      // Loop body
    }
  \end{verbatim}

  Example:
  \lstset{style=CStyle}
  \begin{lstlisting}
#pragma omp parallel for
for (int i = 0; i < N; i++) {
  data[i] = process(i);
}
  \end{lstlisting}

  This is equivalent to a \texttt{parallel} region with a single \texttt{for} loop.
\end{frame}

\begin{frame}[fragile]{Clauses: \texttt{private} and \texttt{shared}}
  Applicable to directives:
  \begin{itemize}
    \item \texttt{parallel}, \texttt{for}, \texttt{parallel for}
  \end{itemize}

  Controls the scope of variables:

  \begin{itemize}
    \item \texttt{shared(var)}: Variable shared among threads (default).
    \item \texttt{private(var)}: Each thread gets its own private copy.
  \end{itemize}

  Example (\texttt{parallel for}):
  \lstset{style=CStyle}
  \begin{lstlisting}
int temp = 0;
#pragma omp parallel for private(temp)
for(int i = 0; i < N; i++) {
    temp = compute(i);
    result[i] = temp;
}
  \end{lstlisting}
\end{frame}

\begin{frame}[fragile]{Clause: \texttt{schedule}}
  Applicable to directives:
  \begin{itemize}
    \item \texttt{for}, \texttt{parallel for}
  \end{itemize}

  Controls iteration distribution among threads:

  Syntax:
  \begin{verbatim}
    schedule(type, chunk_size)
  \end{verbatim}

  Types:
  \begin{itemize}
    \item \texttt{static} (default)
    \item \texttt{dynamic}
    \item \texttt{guided}
  \end{itemize}

  Example (\texttt{parallel for}):
  \lstset{style=CStyle}
  \begin{lstlisting}
#pragma omp parallel for schedule(dynamic,4)
for(int i = 0; i < N; i++) {
  heavy_computation(i);
}
  \end{lstlisting}
\end{frame}

\begin{frame}[fragile]{Clause: \texttt{reduction}}
  Applicable to directives:
  \begin{itemize}
    \item \texttt{parallel}, \texttt{for}, \texttt{parallel for}
  \end{itemize}

  Combines thread results safely into one variable.

  Syntax:
  \begin{verbatim}
    reduction(operator: variable)
  \end{verbatim}

  Common operators: \texttt{+, -, *, max, min}

  Example (\texttt{parallel for}):
  \lstset{style=CStyle}
  \begin{lstlisting}
int total = 0;
#pragma omp parallel for reduction(+:total)
for(int i = 0; i < N; i++) {
    total += array[i];
}
printf("Sum = %d\n", total);
  \end{lstlisting}
\end{frame}

\begin{frame}[fragile]{Clause: \texttt{num\_threads}}
  Applicable to directives:
  \begin{itemize}
    \item \texttt{parallel}, \texttt{parallel for}
  \end{itemize}

  Sets number of threads explicitly:

  Syntax:
  \begin{verbatim}
    num_threads(number_of_threads)
  \end{verbatim}

  Example (\texttt{parallel for}):
  \lstset{style=CStyle}
  \begin{lstlisting}
#pragma omp parallel for num_threads(8)
for(int i = 0; i < N; i++) {
    compute(i);
}
  \end{lstlisting}

  Overrides default thread count and environment settings.
\end{frame}


\begin{frame}[fragile]{Sections}
  Use the \texttt{sections} directive to run independent tasks in parallel:

  \lstset{style=CStyle}
  \begin{lstlisting}
#pragma omp parallel sections
{
  #pragma omp section
  {
    compute_task_A();
  }
  #pragma omp section
  {
    compute_task_B();
  }
  #pragma omp section
  {
    compute_task_C();
  }
}
  \end{lstlisting}
\end{frame}

\section{Synchronization and Data Sharing}
\begin{frame}{Synchronization and Data Sharing}
  \begin{itemize}
    \item Barrier
    \item Critical sections
    \item Atomic operations
    \item Built-in reduction operation
    \item OpenMP locks (similar to mutex)
  \end{itemize}
\end{frame}

\begin{frame}[fragile]{OpenMP Barrier (\texttt{barrier})}
  Synchronizes threads explicitly; threads wait at the barrier until all threads arrive.

  Syntax:
  \begin{verbatim}
    #pragma omp barrier
  \end{verbatim}

  Example:
  \lstset{style=CStyle}
  \begin{lstlisting}
  #pragma omp parallel
  {
      compute_part1();

      #pragma omp barrier  // All threads wait here

      compute_part2();  // Starts only after all threads
                        // finish compute_part1()
  }
  \end{lstlisting}

  Barrier ensures correct sequence in parallel regions.
\end{frame}

\begin{frame}[fragile]{Critical Sections (\texttt{critical})}
  \textbf{Purpose:}
  Ensures only one thread executes a code region at a time, preventing race conditions.

  Syntax:
  \begin{verbatim}
    #pragma omp critical [name]
    {
      // critical section
    }
  \end{verbatim}

  Example:
  \lstset{style=CStyle}
  \begin{lstlisting}
#pragma omp parallel
{
  #pragma omp critical
  {
    sum += compute_value();
  }
}
  \end{lstlisting}

  Usage of this directive ensures safe access to the shared variables within block boundaries.
\end{frame}

\begin{frame}[fragile]{Named Critical Sections}
  Multiple named critical sections prevent unnecessary waiting.

  \textbf{Syntax:}
  \begin{verbatim}
  #pragma omp critical(name)
  {
    // named critical section
  }
  \end{verbatim}

  Example:
  \lstset{style=CStyle}
  \begin{lstlisting}
#pragma omp parallel
{
  #pragma omp critical(update_sum)
  {
    sum += compute_sum();
  }

  #pragma omp critical(update_max)
  {
    max_val = max(max_val, compute_val());
  }
}
  \end{lstlisting}

  Different named critical regions do not block each other.
\end{frame}

\begin{frame}[fragile]{Atomic Operations (\texttt{atomic})}
  Purpose:
  Enforces atomicity of a single memory operation.

  Syntax:
  \begin{verbatim}
  #pragma omp atomic
    expression;
  \end{verbatim}

  Supported operations: \texttt{+, -, *, /, \&, |, \^, ++, --}

  Example:
  \lstset{style=CStyle}
  \begin{lstlisting}
#pragma omp parallel for
for(int i = 0; i < N; i++) {
  #pragma omp atomic
  count += array[i];
}
  \end{lstlisting}

  It is more efficient than \texttt{critical} for simple arithmetic.
\end{frame}

\begin{frame}{\texttt{critical} vs. \texttt{atomic}}
  Key differences between these synchronization methods:

  \begin{itemize}
    \item Critical Sections:
    \begin{itemize}
      \item Allows arbitrary blocks of code.
      \item More general-purpose, but potentially slower due to locking overhead.
    \end{itemize}

    \item Atomic Operations:
    \begin{itemize}
      \item Limited to single, simple memory operations.
      \item Faster, uses hardware-level instructions.
    \end{itemize}
  \end{itemize}

  Use \texttt{atomic} for simple operations, \texttt{critical} for more complex sections.
\end{frame}

\section{OpenMP functions}
\begin{frame}[fragile]{OpenMP Functions: Thread Management}
  Control and query the number of threads.

  Commonly used functions:
  \begin{itemize}
    \item \texttt{omp\_set\_num\_threads(int n)}
    \item \texttt{omp\_get\_num\_threads()}
    \item \texttt{omp\_get\_thread\_num()}
    \item \texttt{omp\_get\_max\_threads()}
  \end{itemize}

  Example:
  \lstset{style=CStyle}
  \begin{lstlisting}
omp_set_num_threads(4);
#pragma omp parallel
{
  int tid = omp_get_thread_num();
  printf("Hello from thread %d\n", tid);
}
  \end{lstlisting}
\end{frame}

\begin{frame}[fragile]{OpenMP Locks}
  Purpose:
  Provide explicit, fine-grained control of synchronization for critical regions.

  API:
  \begin{itemize}
    \item \texttt{omp\_init\_lock()} — Initializes a lock
    \item \texttt{omp\_set\_lock()} — Locks (blocks if unavailable)
    \item \texttt{omp\_unset\_lock()} — Releases a lock
    \item \texttt{omp\_destroy\_lock()} — Frees lock resources
  \end{itemize}

  Example:
  \lstset{style=CStyle}
  \begin{lstlisting}
omp_lock_t lock;
omp_init_lock(&lock);

#pragma omp parallel for
for(int i = 0; i < N; i++) {
    omp_set_lock(&lock);
      sum += compute(i);
    omp_unset_lock(&lock);
}

omp_destroy_lock(&lock);
  \end{lstlisting}

  Explicit locking provides precise synchronization control.
\end{frame}

\begin{frame}[fragile]{OpenMP Functions: Timing}
  Useful functions for measuring execution time:

  \begin{itemize}
    \item \texttt{omp\_get\_wtime()} — returns current time in seconds.
    \item \texttt{omp\_get\_wtick()} — precision of timer.
  \end{itemize}

  Example:
  \lstset{style=CStyle}
  \begin{lstlisting}
double start = omp_get_wtime();

#pragma omp parallel for
for(int i = 0; i < N; i++) {
  heavy_computation(i);
}

double end = omp_get_wtime();
printf("Elapsed time: %f seconds\n", end-start);
  \end{lstlisting}
\end{frame}

\section{Environment variables}

\begin{frame}{Environment Variables in OpenMP}
Environment variables control OpenMP runtime behavior without recompilation.

Common environment variables include:
  \begin{itemize}
    \item \texttt{OMP\_NUM\_THREADS}
    \item \texttt{OMP\_SCHEDULE}
    \item \texttt{OMP\_DYNAMIC}
    \item \texttt{OMP\_NESTED}
  \end{itemize}
\end{frame}

\begin{frame}[fragile]{\texttt{OMP\_NUM\_THREADS}}
  Specifies the default number of threads.

  Example usage:
  \begin{verbatim}
    export OMP_NUM_THREADS=8
    ./my_program
  \end{verbatim}

  Overrides default or explicitly set number of threads within code unless set otherwise by \texttt{num\_threads} clause.
\end{frame}

\begin{frame}[fragile]{\texttt{OMP\_SCHEDULE}}
  Sets default scheduling policy for loops with the \texttt{schedule(runtime)} clause.

  Syntax:
  \begin{verbatim}
    export OMP_SCHEDULE="type,chunk"
  \end{verbatim}

  Example:
  \begin{verbatim}
    export OMP_SCHEDULE="dynamic,4"
    ./my_program
  \end{verbatim}

  Affects loops declared as:
  \begin{verbatim}
    #pragma omp parallel for schedule(runtime)
  \end{verbatim}
\end{frame}

\begin{frame}[fragile]{\texttt{OMP\_DYNAMIC} and \texttt{OMP\_NESTED}}
  Enables dynamic thread adjustment (true/false).

  Example:
  \begin{verbatim}
    export OMP_DYNAMIC=true
  \end{verbatim}
\end{frame}


\begin{frame}[fragile]{\texttt{OMP\_DYNAMIC} and \texttt{OMP\_NESTED}}
  Allows nested parallelism (true/false).

  Example:
  \begin{verbatim}
    export OMP_NESTED=true
  \end{verbatim}

  Nested parallel regions:
  \lstset{style=CStyle}
  \begin{lstlisting}
#pragma omp parallel num_threads(2)
{
  #pragma omp parallel num_threads(2)
  {
    // Nested region, total 4 threads
  }
}
  \end{lstlisting}
\end{frame}

\begin{frame}
  \centering
  \Huge{Thank You!}
\end{frame}

\begin{frame}{References}
  \begin{itemize}
    \item OpenMP Official Specification: \url{https://www.openmp.org/specifications/}
    \item OpenMP Reference Guides: \url{https://www.openmp.org/resources/refguides/}
  \end{itemize}
\end{frame}

\end{document}

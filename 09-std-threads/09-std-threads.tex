\documentclass{beamer}

% Theme choice
\usetheme{Madrid}

% Optional packages
\usepackage{graphicx} % For including images
\usepackage{amsmath}  % For math symbols and formulas
\usepackage{hyperref} % For hyperlinks
\usepackage{tikz}     % For charts
\usepackage{listings}
\usepackage{xcolor}
\usepackage[T1]{fontenc}

\lstdefinestyle{CStyle}{
  language=C,                    % Set the language to C
  basicstyle=\ttfamily\footnotesize\linespread{0.9}\tiny, % Set font style and size
  keywordstyle=\color{blue},      % Color of keywords
  commentstyle=\color{gray},      % Color of comments
  stringstyle=\color{red},        % Color of strings
  showstringspaces=false,         % Do not mark spaces in strings
  breaklines=true,                % Enable line breaks at appropriate places
  breakatwhitespace=false,        % Break lines at any character, not just whitespace
  numbers=left,                   % Show line numbers on the left
  numberstyle=\tiny\color{gray},  % Style for line numbers
  tabsize=4,                      % Set tab width
  keepspaces=true,                % Keep indentation spaces
  frame=single,                   % Add a border around the code
  aboveskip=0pt,                  % Reduce space above the code block
  belowskip=0pt,                  % Reduce space below the code block
  xleftmargin=7.5pt,              % Add left padding (approx. 2.8mm or 10px)
  xrightmargin=15pt,              % Add left padding (approx. 2.8mm or 10px)
}

\AtBeginSection[]{
  \begin{frame}
    \centering
    \Huge\insertsection
  \end{frame}
}

% Title, author, date, and institute (optional)
\title[Parallel Programming. C++ threading]{Parallel Programming course. C++ threading}
\author{Obolenskiy Arseniy, Nesterov Alexander}
\institute{Nizhny Novgorod State University}

\date{\today} % or \date{Month Day, Year}

% Redefine the footline to display both the short title and the university name
\setbeamertemplate{footline}{
  \leavevmode%
  \hbox{%
    \begin{beamercolorbox}[wd=.45\paperwidth,ht=2.5ex,dp=1ex,leftskip=1em,center]{author in head/foot}%
        \usebeamerfont{author in head/foot}\insertshortinstitute % Displays the university name
    \end{beamercolorbox}%
    \begin{beamercolorbox}[wd=.45\paperwidth,ht=2.5ex,dp=1ex,leftskip=1em,center]{author in head/foot}%
      \usebeamerfont{author in head/foot}\insertshorttitle % Displays the short title
    \end{beamercolorbox}%
    \begin{beamercolorbox}[wd=.1\paperwidth,ht=2.5ex,dp=1ex,rightskip=1em,center]{author in head/foot}%
      \usebeamerfont{author in head/foot}\insertframenumber{} / \inserttotalframenumber
    \end{beamercolorbox}}%
  \vskip0pt%
}

\begin{document}

% Title slide
\begin{frame}
  \titlepage
\end{frame}

% Table of Contents (optional)
\begin{frame}{Contents}
  \tableofcontents
\end{frame}

\section{Introduction to C++ threading API}

\begin{frame}{What is \texttt{std::thread}?}
  \begin{itemize}
    \item Part of the C++11 thread support library (\texttt{<thread>}, \texttt{<mutex>}, etc.)
    \item Low-level, manual thread creation and management
    \item Relies on the OS native threads under the hood
    \item Provides:
      \begin{itemize}
        \item \texttt{std::thread} for launching threads
        \item Synchronization primitives (\texttt{std::mutex}, \texttt{std::condition\_variable}, \texttt{std::atomic})
        \item Utilities for futures and promises (\texttt{std::future}, \texttt{std::promise})
      \end{itemize}
  \end{itemize}
\end{frame}

\begin{frame}{Threading before C++11}
  \begin{itemize}
    \item POSIX threads (\texttt{pthread}): C-based API, widely used on Unix-like systems
    \item Windows threads: Win32 API with \texttt{CreateThread}, \texttt{CRITICAL\_SECTION}, and events
    \item External libraries (e.g. Boost.Thread)
  \end{itemize}
  Challenges and inconveniences:
  \begin{itemize}
    \item non-standard APIs (towards C++)
    \item platform dependent APIs
    \item verbose initialization boilerplate
    \item manual resource management
  \end{itemize}
\end{frame}

\begin{frame}{\texttt{std::thread} history}
  \begin{itemize}
    \item C++11 (2011): Introduced \texttt{std::thread}, \texttt{std::mutex}, \texttt{std::future}, etc.
    \item C++14/17/20: Incremental improvements (e.g., \texttt{std::hardware\_constructive\_interference\_size})
    \item Widely adopted as the base for custom thread pools and concurrency utilities
    \item Now the foundation of many higher-level C++ concurrency libraries
  \end{itemize}
\end{frame}

\begin{frame}{OpenMP vs TBB vs \texttt{std::thread}}
  \begin{itemize}
    \item \textbf{OpenMP}
      \begin{itemize}
        \item Directive-based, compiler-driven shared-memory parallelism
        \item Simple loop and region parallelism
      \end{itemize}
    \item \textbf{TBB}
      \begin{itemize}
        \item Template library, task-based parallelism with work-stealing
        \item High-level parallel patterns (\texttt{tbb::parallel\_for}, \texttt{tbb::parallel\_reduce}, etc.)
      \end{itemize}
    \item \textbf{std::thread}
      \begin{itemize}
        \item Low-level manual thread API
        \item Full control over thread lifetime, but more boilerplate
        \item Foundation for building custom task systems or thread pools
      \end{itemize}
  \end{itemize}
\end{frame}

\begin{frame}{Pros and cons of \texttt{std::thread}}
  \begin{columns}
    \column{0.5\textwidth}
      \textbf{Pros}
      \begin{itemize}
        \item Complete control over thread creation and destruction
        \item No hidden scheduler—behavior is predictable
        \item Part of standard C++, portable across platforms
        \item Good for learning fundamentals
      \end{itemize}
    \column{0.5\textwidth}
      \textbf{Cons}
      \begin{itemize}
        \item Manual management—risk of leaks, detach/join errors
        \item Verbose boilerplate for synchronization
        \item No built-in task scheduling or work-stealing
        \item Harder to scale and tune compared to higher-level APIs
      \end{itemize}
  \end{columns}
\end{frame}

\section{C++ STL threading API (\texttt{std::thread}, \texttt{std::jthread})}

\begin{frame}{\texttt{std::thread} details}
  \begin{itemize}
    \item Constructors:
      \begin{itemize}
        \item \texttt{std::thread(callable, args...)}: starts a new thread executing \texttt{callable} with \texttt{args}
        \item Default constructor: creates a thread object without an associated thread
      \end{itemize}
    \item Member functions:
      \begin{itemize}
        \item \texttt{join()}: blocks until the thread finishes execution
        \item \texttt{detach()}: detaches the thread to run independently
        \item \texttt{joinable()}: checks whether the thread can be joined
        \item \texttt{get\_id()}: returns the \texttt{std::thread::id} of the thread
      \end{itemize}
    \item Properties:
      \begin{itemize}
        \item Move-only: supports move construction and assignment; copy operations are deleted
        \item Static \texttt{hardware\_concurrency()}: returns the number of concurrent threads supported
      \end{itemize}
  \end{itemize}
\end{frame}

\begin{frame}[fragile]{Creating and launching threads}
  \lstset{style=CStyle}
  \begin{lstlisting}
#include <thread>
#include <iostream>

void worker(int id) {
  std::cout << "Worker " << id << " running\n";
}

int main() {
  // Launch two threads with argument
  std::thread t1(worker, 1);
  std::thread t2(worker, 2);

  // Wait for them to finish
  if (t1.joinable()) {
    t1.join();
  }
  if (t2.joinable()) {
    t2.join();
  }

  return 0;
}
  \end{lstlisting}
  \begin{itemize}
    \item Construct \texttt{std::thread} with a callable + optional args
    \item Must call \texttt{join()} or \texttt{detach()} before destruction
    \item Threads are move-only: no copy construction/assignment
  \end{itemize}
\end{frame}

\begin{frame}[fragile]{Managing thread lifetime}
  \begin{itemize}
    \item \texttt{join()}: waits for thread completion
    \item \texttt{detach()}: allows thread to run independently (daemon-like)
  \end{itemize}
  \lstset{style=CStyle}
  \begin{lstlisting}
std::thread t(worker);
// ...
if (t.joinable()) {
    t.join();
}
  \end{lstlisting}
  \begin{itemize}
    \item \texttt{joinable()}: check if thread can be joined
  \end{itemize}
  \begin{itemize}
    \item Detached threads continue after \texttt{main}—dangerous if resources go out of scope
    \item Always ensure each thread is either joined or detached
  \end{itemize}
\end{frame}

\begin{frame}[fragile]{Passing arguments to threads}
  \lstset{style=CStyle}
  \begin{lstlisting}
#include <thread>
#include <string>
#include <iostream>

void greet(const std::string &name) {
  std::cout << "Hello, " << name << "!\n";
}

int main() {
  std::string user = "Alice";
  // Pass by reference: use std::ref
  std::thread t(greet, std::ref(user));
  t.join();
  return 0;
}
  \end{lstlisting}
  \begin{itemize}
    \item By default, args are copied into the new thread
    \item Use \texttt{std::ref()} to pass references
  \end{itemize}
\end{frame}

\begin{frame}{Querying hardware concurrency capabilities}
  \begin{itemize}
    \item \texttt{std::thread::hardware\_concurrency()} returns the number of supported threads
    \item May potentially return 0
  \end{itemize}
  \texttt{unsigned n = std::thread::hardware\_concurrency();}
  Usage example: to size thread pools or partition work
\end{frame}

\begin{frame}[fragile]{\texttt{std::jthread}}
  \begin{itemize}
    \item Introduced in C++20 as a safer and more convenient alternative to \texttt{std::thread}.
    \item Automatically joins upon destruction, reducing the risk of detached threads.
    \item Supports cooperative cancellation via \texttt{std::stop\_token}.
  \end{itemize}
  Example usage:
  \lstset{style=CStyle}
  \begin{lstlisting}
#include <iostream>
#include <thread>
#include <chrono>

void task(std::stop_token stoken) {
  while (!stoken.stop_requested()) {
    std::cout << "Working...\n";
    std::this_thread::sleep_for(std::chrono::milliseconds(100));
  }
  std::cout << "Task stopping.\n";
}

int main() {
  std::jthread jt(task);
  std::this_thread::sleep_for(std::chrono::seconds(1));
  // Request stop automatically during destruction or manually via jt.request_stop();
  return 0;
}
  \end{lstlisting}
\end{frame}

\section{\texttt{std::future}, \texttt{std::promise} and \texttt{std::async}}

\begin{frame}{Future/Promise and Async Execution}
  \begin{itemize}
    \item Future and promise provide a mechanism for asynchronous communication between threads
    \item A promise allows one thread to set a value or report an error
    \item The associated future retrieves the value, waiting if necessary
    \item std::async simplifies launching asynchronous tasks without explicit thread management
    \item These features promote decoupling of computation and enhance concurrency control
  \end{itemize}
\end{frame}

\begin{frame}[fragile]{\texttt{std::future} and \texttt{std::promise}}
  \lstset{style=CStyle}
  \begin{lstlisting}
#include <future>
#include <iostream>

int compute() {
  int result;
  ...  // heavy computation
  return result;
}

int main() {
  std::promise<int> prom;
  std::future<int> fut = prom.get_future();

  std::thread t([&prom]{
      prom.set_value(compute());
  });

  std::cout << "Result: " << fut.get() << "\n";
  t.join();
  return 0;
}
  \end{lstlisting}

  \texttt{std::promise} sets a value (or failure)

  \texttt{std::future} retrieves it on demand
\end{frame}

\begin{frame}[fragile]{Using \texttt{std::async}}
  \begin{itemize}
    \item Launches tasks asynchronously and returns a \texttt{std::future}
    \item Can run immediately in a new thread or be deferred until needed
    \item Simplifies asynchronous programming by handling thread management
  \end{itemize}
  \lstset{style=CStyle}
  \begin{lstlisting}
#include <future>
#include <iostream>

int computeSquare(int x) {
  // Simulate heavy computation
  std::this_thread::sleep_for(std::chrono::seconds(1));
  return x * x;
}

int main() {
  // Launch computeSquare asynchronously. Using std::launch::async forces immediate execution
  std::future<int> fut = std::async(std::launch::async, computeSquare, 5);
  std::cout << "Square: " << fut.get() << std::endl;
  return 0;
}
  \end{lstlisting}
\end{frame}

\begin{frame}[fragile]{Different launch policies available in \texttt{std::async}}
  \lstset{style=CStyle}
  \begin{lstlisting}
#include ...

int computeCube(int x) { ... }

int main() {
  // Using deferred launch: execution is postponed until get() is called
  std::future<int> futDeferred = std::async(std::launch::deferred, computeCube, 3);
  std::cout << "Deferred result: " << futDeferred.get() << std::endl;
  
  // Using async launch: task is executed immediately in a new thread
  std::future<int> futAsync = std::async(std::launch::async, computeCube, 3);
  std::cout << "Async result: " << futAsync.get() << std::endl;
  
  // Using default launch policy: implementation defined behavior
  std::future<int> futDefault = std::async(computeCube, 3);
  std::cout << "Default launch result: " << futDefault.get() << std::endl;
  
  return 0;
}
  \end{lstlisting}
  Policies:
  \begin{itemize}
    \item \texttt{std::launch::deferred}: Execution is delayed until \texttt{get()} is called
    \item \texttt{std::launch::async}: Execution starts immediately in a separate thread
    \item Default policy: The decision is left to the implementation
  \end{itemize}
\end{frame}

\section{Synchronization primitives (mutexes, condition variables, \dots)}

\begin{frame}{Synchronization primitives overview}
  \begin{itemize}
    \item C++ provides several mechanisms to coordinate concurrent operations:
      \begin{itemize}
        \item Mutual exclusion: \texttt{std::mutex}, \texttt{std::lock\_guard}, \texttt{std::unique\_lock}
        \item Condition variables: \texttt{std::condition\_variable}
        \item Atomic operations: \texttt{std::atomic}
      \end{itemize}
    \item Choose the appropriate primitive based on the required control and performance
    \item Proper synchronization is key to ensuring thread safety and avoiding data races
  \end{itemize}
\end{frame}

\begin{frame}[fragile]{\texttt{std::mutex}}
  \lstset{style=CStyle}
  \begin{lstlisting}
#include <mutex>
#include <thread>
#include <iostream>

std::mutex mtx;
int counter = 0;

void increment() {
  mtx.lock();
  {
    ++counter; // protected section
  }
  mtx.unlock();
}

int main() {
  std::thread t1(increment);
  std::thread t2(increment);

  t1.join();
  t2.join();

  std::cout << "Final counter value: " << counter << std::endl;
  return 0;
}
  \end{lstlisting}
  \begin{itemize}
    \item Manual locking with \texttt{mtx.lock()} and unlocking with \texttt{mtx.unlock()}
    \item Be cautious with exceptions to avoid deadlocks
  \end{itemize}
  There is a way to ensure that mutex will be unlocked\dots
\end{frame}

\begin{frame}[fragile]{\texttt{std::mutex} and \texttt{std::lock\_guard}}
  \lstset{style=CStyle}
  \begin{lstlisting}
#include <mutex>

std::mutex mtx;
int counter = 0;

void increment() {
  std::lock_guard<std::mutex> lock(mtx);
  ++counter;  // protected section
}
  \end{lstlisting}
  \begin{itemize}
    \item \texttt{std::mutex}: exclusive lock
    \item \texttt{std::lock\_guard}: RAII wrapper—locks on construction, unlocks on destruction
  \end{itemize}
\end{frame}

\begin{frame}[fragile]{\texttt{std::unique\_lock} and \texttt{std::condition\_variable}}
  \lstset{style=CStyle}
  \begin{lstlisting}
#include <mutex>
#include <condition_variable>

std::mutex mtx;
std::condition_variable cv;
bool ready = false;

void worker() {
  std::unique_lock<std::mutex> lock(mtx);
  cv.wait(lock, []{ return ready; });
  // proceed once ready == true
}

void notifier() {
  {
    std::lock_guard<std::mutex> lock(mtx);
    ready = true;
  }
  cv.notify_one();
}
  \end{lstlisting}
  \texttt{std::unique\_lock}: provides flexible locking mechanisms such as deferred locking, timed locking, and manual unlocking/relocking, unlike \texttt{lock\_guard} which locks immediately and strictly scopes the lock.
  \texttt{std::condition\_variable}: allows one or more threads to wait until a particular condition is met. It’s typically used with \texttt{unique\_lock} to enable the thread to wait and then be notified.
\end{frame}

\begin{frame}{\texttt{std::atomic}}
  Lock-free primitives for simple data types. Avoid mutex overhead for simple counters and flags
  Supported operations list:
  \begin{itemize}
    \item \texttt{store}: assign a new value
    \item \texttt{load}: retrieve the current value
    \item \texttt{exchange}: replace the value and obtain the old value
    \item \texttt{compare\_exchange\_weak}/\texttt{compare\_exchange\_strong}: atomically compare and set
    \item \texttt{fetch\_add}: add to the value and return the previous value
    \item \texttt{fetch\_sub}: subtract from the value and return the previous value
    \item \texttt{fetch\_and}: perform bitwise AND and return the previous value
    \item \texttt{fetch\_or}: perform bitwise OR and return the previous value
    \item \texttt{fetch\_xor}: perform bitwise XOR and return the previous value
  \end{itemize}
\end{frame}

\begin{frame}[fragile]{Atomic operations usage example}
  \lstset{style=CStyle}
  \begin{lstlisting}
#include <atomic>
#include <thread>
#include <iostream>

std::atomic<int> counter(0);

void increment() {
  for (int i = 0; i < 100000; ++i) {
    counter.fetch_add(1, std::memory_order_relaxed);
  }
}

int main() {
  std::thread t1(increment);
  std::thread t2(increment);

  t1.join();
  t2.join();

  std::cout << "Final counter value: " << counter.load() << std::endl;
  return 0;
}
  \end{lstlisting}
  \texttt{std::atomic} provides lock-free operations. Use \texttt{fetch\_add} method and load to operate on atomic variables safely
\end{frame}

\section{Best practices and recommendations}

\begin{frame}{Best practices and recommendations}
  \begin{itemize}
    \item Always join or detach threads before destruction
    \item Prefer RAII wrappers (\texttt{std::lock\_guard}, \texttt{std::unique\_lock})
    \item Minimize shared state; prefer message passing or futures
    \item Be cautious with detached threads, manage lifetimes carefully
    \item Consider higher-level thread pools (e.g., \texttt{std::async})
  \end{itemize}
\end{frame}

\begin{frame}{When and where to use \texttt{std::thread} in real world scenarios}
  \begin{itemize}
    \item Fine-grained control over threading
    \item Building custom schedulers or thread pools
    \item Interfacing directly with OS thread APIs
    \item Performance critical sections where you avoid scheduler overhead
    \item When introducing 3rdparty library is an overkill
  \end{itemize}
\end{frame}

\begin{frame}
  \centering
  \Huge{Thank You!}
\end{frame}

\begin{frame}{References}
  \begin{itemize}
    \item cppreference.com: \url{https://en.cppreference.com/w/cpp/thread}
  \end{itemize}
\end{frame}

\end{document}
